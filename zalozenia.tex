\documentclass[11pt]{article}
 
\usepackage[T1]{fontenc}
\usepackage[polish]{babel}
\usepackage[utf8]{inputenc}
\usepackage{lmodern}
\selectlanguage{polish}
\title{Założenia projektu środowiska projektu z przedmiotu Sztuczna Inteligencja}
\author{
  Gawroński Kacper
  \and
  Nowicka Julia
}


\begin{document}
\maketitle
\section{Opis problemu.}
Dana jest macierz $M_{m,n}$. Wśród m-1 wierszy około 10\% przyjmuje wartość 1, pozostałe wartość 0.\\
Wiersz m przyjmuje wartości początkowe równe 0, oraz jest jednocześnie "kuchnią" - miejscem, do którego "kelner" przynosi zamówienia oraz z którego odbiera zrealizowane zamówienia.\\
Pozycja "kelnera" jest opisana jako para liczb (a,b) z zakresów $0\le a \le n$ oraz $0\le b \le m$ oznaczająca jego aktualną pozycję w macierzy.\\
Ponadto z "kelnerem" powiązana jest lista zawierająca "zamówienia", które są do zrealizowania w kuchni.\\
"Kuchnia" przyjmuje zlecenia od "kelnera", oraz je realizuje - co jeden ruch "kelnera" zmniejszany jest czas pozostały do ralizacji każdego "zamówienia".\\
Gdy czas "zamówienia" w kuchni osiągnie wartość mniejszą bądź równą zero, "zamówienie" przechodzi do listy odpowiadającej zrealizowanych przez kuchnię "zamówień".\\
Kelner wkraczając na obszar kuchni automatycznie przejmuje wszystkie zrealizowane "zamówienia" dodając je do przechowywanej listy oraz może je dostarczyć do "klienta".\\
Definicje czasu potrzebnego do realizacji konkretnego zamówienia są zdefiniowane w słowniku zamówień.\\
Wartość 1 w macierzy definiuje jedynie pozycję klienta, nie "zamówienie". "Zamówienie" jest ustalane losowo po wejściu "kelnera" na pozycję klienta oraz przeczekanie jednej tury.\\
Zamówienia są przechowywane w macierzy rzadkiej jako lista pozycji wraz z zamówieniem.\\
Środowisko powinno przeprowadzać podstawową walidację, tj. czy "kelner" nie wychodzi poza dopuszczalny zakres ruchów, oraz czy dostarczył prawidłową realizację zamówienia do klienta.\\
\paragraph{Cel}
Celem jest opracowanie rozwiązania zebrania zamówień od wszystkich klientów oraz dostarczenia im realizacji zamówień, w możliwie krótkim czasie.
\paragraph{Komunikacja}
Środowisko powinno umożliwiać komunikację przetwarzając ciąg liter ze zbioru \{N,S,E,W\} odpowiadającym kierunkom oraz litery H odpowiadającej przeczekaniu tury.\\

\end{document}
