\documentclass[11pt]{article}
 
\usepackage[T1]{fontenc}
\usepackage[polish]{babel}
\usepackage[utf8]{inputenc}
\usepackage{lmodern}
\selectlanguage{polish}
\title{Założenia projektu środowiska projektu z przedmiotu Sztuczna Inteligencja}
\author{
  Gawroński Kacper
  \and
  Nowicka Julia
}


\begin{document}
\maketitle
\section{Opis problemu.}
Dana jest macierz $M_{m,n}$. Wśród m-1 wierszy około 10\% przyjmuje wartość 1, pozostałe wartość 0.\\
Wiersz m przyjmuje wartości początkowe równe 0, oraz jest jednocześnie \textit{kuchnią} - miejscem, do którego \textit{kelner} przynosi zamówienia oraz z którego odbiera zrealizowane zamówienia.\\
Pozycja \textit{kelnera} jest opisana jako para liczb (a,b) z zakresów $0\le a \le n$ oraz $0\le b \le m$ oznaczająca jego aktualną pozycję w macierzy.\\
Ponadto z \textit{kelnerem} powiązana jest lista zawierająca \textit{zamówienia}, które są do zrealizowania w kuchni.\\
\textit{Kuchnia} przyjmuje zlecenia od \textit{kelnera}, oraz je realizuje - co jeden ruch \textit{kelnera} zmniejszany jest czas pozostały do ralizacji każdego \textit{zamówienia}.\\
Gdy czas \textit{zamówienia} w kuchni osiągnie wartość mniejszą bądź równą zero, \textit{zamówienie} przechodzi do listy odpowiadającej zrealizowanych przez kuchnię \textit{zamówień}.\\
Kelner wkraczając na obszar kuchni automatycznie przejmuje wszystkie zrealizowane \textit{zamówienia} dodając je do przechowywanej listy oraz może je dostarczyć do \textit{klienta}.\\
Definicje czasu potrzebnego do realizacji konkretnego \textit{zamówienia} są zdefiniowane w słowniku zamówień.\\
Wartość 1 w macierzy definiuje jedynie pozycję klienta, nie \textit{zamówienie}. \textit{Zamówienie} jest ustalane losowo po wejściu \textit{kelnera} na pozycję klienta oraz przeczekanie jednej tury.\\
Zamówienia są przechowywane w macierzy rzadkiej jako lista pozycji wraz z \textit{zamówieniem}.\\
Środowisko powinno przeprowadzać podstawową walidację, tj. czy \textit{kelner} nie wychodzi poza dopuszczalny zakres ruchów, oraz czy dostarczył prawidłową realizację zamówienia do klienta.\\
\paragraph{Cel}
Celem jest opracowanie rozwiązania zebrania zamówień od wszystkich klientów oraz dostarczenia im realizacji \textit{zamówień}, w możliwie krótkim czasie.
\paragraph{Komunikacja}
Środowisko powinno umożliwiać komunikację przetwarzając ciąg liter ze zbioru \{N,S,E,W\} odpowiadającym kierunkom oraz litery H odpowiadającej przeczekaniu tury bądź akcji (oddania/pobrania \textit{zamówień}).\\

\end{document}
